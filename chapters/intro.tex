% Set your chapter name here
\chapter{My first chapter}

Here is some introductory statement.

\section{The Riemann hypothesis}

In this section, we give a long-awaited proof of the Riemann hypothesis.

\subsection{Preliminaries}

\begin{definition}
    The function $\zeta: \{s \in \bb{C}: \Re(s) > 1\} \to \bb{C}$ such that
    \[
        \zeta(s)
        = \sum_{n = 1}^{\infty} \frac{1}{n^s}
        = \prod_{p \text{ prime}} \frac{1}{1 - p^{-s}}.
    \]
    is called the \emph{Riemann zeta function}.
\end{definition}

Considering the analytic continuation of $\zeta$, Riemann showed that $\zeta$ satisfied a particular meromorphic functional equation:

\begin{theorem}
\label{thm_rz_func_eq}
    For \emph{all} $s \in \bb{C}$, $\zeta$ satisfies
    \[
        \zeta(s) = 2^s \pi^{s - 1} \sin\left(\frac{\pi s}{2}\right) \Gamma(1 - s) \zeta(1 - s).
    \]
\end{theorem}

\subsection{Zeroes of the Riemann zeta function: the hypothesis}

A consequence of \cref{thm_rz_func_eq} is that every negative, even integer is a zero of $\zeta$. (The same is \emph{not} true of the positive, even integers: why?) Each of these zeroes is called a \emph{trivial zero}. A natural question for one to ask is where, if any, the \emph{nontrivial zeroes} of $\zeta$ lie. Some of these have been found, and curiously they all lay on the strip $\Re(s) = \frac{1}{2}$. This curiosity led Riemann, in 1859, to conjecture that

\begin{block}
    \begin{conjecture}[Riemann, 1859]
        All nontrivial zeroes of the Riemann zeta function lie on the strip $Re(s) = \frac{1}{2}$.
    \end{conjecture}
\end{block}

We can now prove this conjecture holds, turning it into a theorem:

\begin{proof}
    Obvious.
\end{proof}
